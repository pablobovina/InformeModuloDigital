\section{Submodulo DDS2}

El DDS2 es un generador de señales con un rango de frecuencia de 0 a 200 Mhz.

El usuario puede almacenar hasta 2 frecuencias y un total de 16 valores de fase
en la memoria para combinarlos en la ejecucion de un programa.

Tambien hay una relacion entre el PP2 y el DDS2, puesto que este tiene como entrada
los Pulsos 8,9,10,11,12,13,14 para su manipulacion externa por aquel durante
la ejecucion de un programa.

\subsection{almacenamiento de fases}

El registro de RAM por fase es de 2 btyes, pero el valor de fase es un entero de 14 bits,
por lo tanto los 4 bits mas significativos son ignorados.

El MSB de una fase debe almacenarse siempre en una direccion par de la RAM, luego
el LSB de la misma en el valor impar siguiente contiguo.

Antes de almacenar el valor entero de la fase F debemos convertirlo a su equivalente
en unidades de la siguiente manera:
\noindent
\begin{gather}
        F_h \in \mathbb{N} \\
        F_l \in \mathbb{N} \\
        F \in \mathbb{N} \\
        F_h = (45 * F) / 256 \\          
        F_l = (45 * F) - (F_h * 256)
\end{gather}




\subsection{almacenamiento de frecuencias}



\subsection{direccionamiento de fases}
Se admiten hasta 16 valores de fase los cuales son direccionados con 4 bits
correspondientes a los pulsos 11,12,13,14.

\begin{itemize}
    \item modo carga de fases habilitada
    \item direccionamiento con pulsos 11,12,13,14
    \item pulso 10
    \item pulso 9
\end{itemize}

\subsection{direccionamiento de frecuencias}
Se admiten hasta 2 frecuencias de trabajo, se seleccionan con el pulso 8.




\begin{table}[ht]
    \centering
    \begin{tabular}{|l|l|l|l|}
    \hline
    Direccion  & Descripcion             & Modo      \\
    \hline
    0x70       & direccionamiento        & Escritura \\
    \hline
    0x71       & modo                    & Escritura \\
    \hline
    0x72       & reset                   & Escritura \\
    \hline
    0x73       & test                    & Lectura   \\
    \hline
    0x74       & se\~nal de escritura      & Escritura \\
    \hline
    0x75       & direccionamiento        & Escritura \\
    \hline
    0x76       & se\~nal de transferencia  & Escritura \\
    \hline
    0x77       & test                    & Lectura   \\
    \hline
    0x78       & se\~nal de escritura      & Escritura \\
    \hline
\end{tabular}
\caption{\label{tab:tableTestCases}Resgistros internos del DDS2.}
\end{table}


\newpage
