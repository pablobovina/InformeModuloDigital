\section{Subm\'odulo DDS2}

El subm\'odulo DDS2 es un generador de se\~nales digitales con un rango de frecuencia de 0 a 80 Mhz.
El usuario puede almacenar hasta 2 frecuencias y un total de 16 fases para 
combinarlos en un experimento.
Existe una relaci\'on entre el PP2 y el DDS2, puesto que este tiene como entrada
los pulsos 8,9,10,11,12,13,14 para su manipulaci\'on externa por aquel durante
la ejecuci\'on de un programa.

\subsection{Configuraci\'on inicial, activaci\'on y desactivaci\'on.}

\begin{table}[ht]
    \centering
    \begin{tabular}{|l|l|l|l|l|l|l|l|l|l|l|}
    \hline
    Direcci\'on  &B7 & B6 & B5 & B4& B3 & B2& B1 & B0 &Hex\\
    \hline
    1D  &0 & 0 & 0 & 1& 1 & 1& 1 & 1 &0x17\\
    \hline
    1E  &0 & 1 & 0 & 0& 0 & 1& 0 & 0 &0x44\\
    \hline
    1F  &0 & 0 & 0 & 0& 0 & 0& 1 & 0 & 0x02\\
    \hline
    20  &0 & 0 & 0 & 0 & 0 & 0& 0 & 0&0x00 \\
    \hline
    \end{tabular}
    \caption{\label{tab:dds2_config_inicial}Reset}
    \end{table}

    \begin{table}[ht]
    \centering
    \begin{tabular}{|l|l|l|l|l|l|l|l|l|l|l|}
    \hline
    Direcci\'on  &B7 & B6 & B5 & B4& B3 & B2& B1 & B0 &Hex\\
    \hline
    1D  &0 & 0 & 0 & 1& 0 & 0& 0 & 0 &0x10\\
    \hline
    1E  &0 & 1 & 0 & 0& 0 & 1& 0 & 0 &0x44\\
    \hline
    1F  &0 & 0 & 0 & 0& 0 & 0& 1 & 0 & 0x02\\
    \hline
    20  &0 & 0 & 0 & 0 & 0 & 0& 0 & 0&0x00 \\
    \hline
    \end{tabular}
    \caption{\label{tab:dds2_activate}Activaci\'on}
    \end{table}
    
    \begin{table}[ht]
    \centering
    \begin{tabular}{|l|l|l|l|l|l|l|l|l|l|l|}
    \hline
    Direcci\'on  &B7 & B6 & B5 & B4& B3 & B2& B1 & B0 &Hex\\
    \hline
    1D  &0 & 0 & 0 & 1& 1 & 1& 1 & 1 &0x17\\
    \hline
    1E  &0 & 1 & 0 & 0& 0 & 1& 0 & 0 &0x44\\
    \hline
    1F  &0 & 0 & 0 & 0& 0 & 0& 1 & 0 & 0x02\\
    \hline
    20  &0 & 0 & 0 & 0 & 0 & 0& 0 & 0&0x00 \\
    \hline
    \end{tabular}
    \caption{\label{tab:dds2_deactivate}Desactivaci\'on}
    \end{table}
\newpage
\subsection{Registros del DDS2}
\begin{table}[ht]
    \centering
    \begin{tabular}{|l|l|l|l|}
    \hline
    Direcci\'on  & Descripci\'on             & Modo      \\
    \hline
    0x70       & direccionamiento        & Escritura \\
    \hline
    0x71       & modo                    & Escritura \\
    \hline
    0x72       & reset                   & Escritura \\
    \hline
    0x73       & test                    & Lectura   \\
    \hline
    0x74       & se\~nal de escritura      & Escritura \\
    \hline
    0x75       & direccionamiento        & Escritura \\
    \hline
    0x76       & se\~nal de transferencia  & Escritura \\
    \hline
    0x77       & test                    & Lectura   \\
    \hline
    0x78       & se\~nal de escritura      & Escritura \\
    \hline
\end{tabular}
\caption{\label{tab:registros_internos_dds2}Registros internos del DDS2.}
\end{table}


\subsection{Fases}

El DDS2 tiene disponible 32 posiciones de memoria 8 bits para el almacenamiento de fases
con un bus de datos de 8 bits.
La fase es un valor de 14 bits, los 2 bits restantes del MSB son descartados.
El MSB de una fase debe almacenarse siempre en una direcci\'on par de la memoria, luego
el LSB de la misma en el valor impar contiguo siguiente.
Antes de almacenar el valor entero de la fase F debemos convertirlo a su equivalente
en unidades de 45 de la siguiente manera:
\begin{gather}
        F_h \in \mathbb{N} \\
        F_l \in \mathbb{N} \\
        F \in \mathbb{N} \\
        F_h = (45 * F) / 256 \\          
        F_l = (45 * F) - (F_h * 256)
\end{gather}

\begin{table}[ht]
    \centering
    \begin{tabular}{|l|l|l|}
    \hline
    Direcciones de RAM disponibles           & Modo \\
    \hline
    0x00 0x02 0x04 0x06 0x08 0x0A 0x0C 0x0E  & 0x02 \\
    \hline
    0x10 0x12 0x14 0x016 0x18 0x1A 0x1C 0x1E & 0x02 \\
    \hline
\end{tabular}
\caption{\label{tab:tableTestCases}Direcciones de Fase.}
\end{table}

\subsubsection{Algoritmo base de almacenamiento de fases}
\begin{algorithm}[H]
    \caption{Almacenamiento de una fase en direcci\'on 0x00}\label{algo_phases}
    \begin{algorithmic}[1]
    \Procedure{SavePhase}{F}
    \State // {MSB y LSB de valor de fase}
    \State $F_h = (45 * F) / 256$         
    \State $F_l = (45 * F) - (F_h * 256)$
    \State // {direcciones contiguas}
    \State $dir_h \gets 0x00$
    \State $dir_l \gets dir_h + 1$
    \State // {Modo carga de fases}
    \State $write(0x71, 0x02)$
    \State // {MSB}
    \State $write(0x70, dir_h)$ 
    \State $write(0x74, f_h)$
    \State // {LSB}
    \State $write(0x70, dir_l)$
    \State $write(0x74, f_l)$
    \State // {Modo PC}
    \State $write(0x71, 0x00)$
    \EndProcedure
    \end{algorithmic}
\end{algorithm}

\subsubsection{Direccionamiento de fases}
Los 16 valores de fase son direccionados con 4 pulsos correspondientes a los 
pulsos 11,12,13,14. Con el pulso 9 se activa la carga de fases y con el pulso 10
se transfiere la fase direccionada al registro de trabajo.
El tiempo entre el pulso 9 y 10 debe ser menor a 100 nanosegundos.

\subsection{Frecuencias}

El DDS2 tiene disponible 12 registros internos de frecuencia de 8 bits, cada frecuencia ocupa 6 registros, 
por lo tanto se pueden almacenar 2 frecuencias.
Las frecuencias disponibles van de 0 a 80 Mhz y el valor que se almacena en los registros
en representaci\'on se obtiene con la siguiente c\'alculo:
\noindent
\begin{gather}
    clock = 2 \times 10^{6} \\
    frec \in \mathbb{N} \land 0 < frec < clock\\
    valor = frec \times (2^{48} -1 ) / clock
\end{gather}

\begin{table}[ht]
    \centering
    \begin{tabular}{|l|l|l|l|}
    \hline
    Nro. Frecuencia    & Registros       & Modo \\
    \hline
     1 & 0x04 0x05 0x06 0x07 0x08 0x09   & 0x00 \\
    \hline
     2 & 0x0A 0x0B 0x0C 0x0D 0x0E 0x0F   & 0x00 \\
    \hline
\end{tabular}
\caption{\label{tab:registros_frec}Registros de Frecuencia.}
\end{table}

\subsubsection{Algoritmo base de almacenamiento de frecuencias}
\begin{algorithm}[H]
    \caption{Almacenamiento de una frecuencia de trabajo 1.}\label{algo_frec}
    \begin{algorithmic}[1]
    \Procedure{SaveFrequency}{F}
    \State // {Conversi\'on de la frecuencia deseada al valor}
    \State $clock \gets 2 \times 10^{6}$
    \State $value = frec \times (2^{48} -1 ) / clock $
    \State // {Modo PC}
    \State $write(0x71, 0x00)$
    \State // {primero MSB hasta LSB}
    \State // {Byte 5}
    \State $write(0x75, 0x04)$ 
    \State $write(0x78, value_5)$
    \State // {Byte 4}
    \State $write(0x75, 0x05)$ 
    \State $write(0x78, value_4)$
    \State // {Byte 3}
    \State $write(0x75, 0x06)$ 
    \State $write(0x78, value_3)$
    \State // {Byte 2}
    \State $write(0x75, 0x07)$ 
    \State $write(0x78, value_2)$
    \State // {Byte 1}
    \State $write(0x75, 0x08)$ 
    \State $write(0x78, value_1)$
    \State // {Byte 0}
    \State $write(0x75, 0x09)$ 
    \State $write(0x78, value_0)$
    \State // {Actualizaci\'on registro de trabajo}
    \State $write(0x76, 0x00)$
    \EndProcedure
    \end{algorithmic}
\end{algorithm}

\subsubsection{Direccionamiento de frecuencias}
Se admiten hasta 2 frecuencias de trabajo, se seleccionan con el pulso 8.

\newpage
