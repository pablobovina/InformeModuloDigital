\section{Definici\'on de un experimento}

Para representar la definici\'on de un experimento se eligi\'o notaci\'on JSON, 
por el soporte built-in por parte de los lenguajes usados para desarrollar el sistema, 
Python y Javascript, tambi\'en por la implementaci\'on de servicios REST necesarios 
en el \textit{Servidor}\cite{json_standar}.
Haciendo uso de aritm\'etica podemos deducir que la representaci\'on de un experimento en JSON
es liviana puesto que el n\'umero de instrucciones de un programa \(P\) ser\'a de \(N < 512 \) 
y para un archivo de texto de 512 l\'ineas y 80 columnas de caracteres de 1 Byte 
tiene un peso aproximado de 327.68 Kilobytes.
La validaci\'on del esquema JSON de un experimento, es llevada a cabo por un m\'odulo programado
dentro del sistema para tal fin.
Cabe destacar que los tipos de datos presentes en la especificaci\'on de un experimento son
soportados sin p\'erdida de precisi\'on por parte de la notaci\'on JSON.\cite{json_ref}

\lstset{
    string=[s]{"}{"},
    stringstyle=\color{blue},
    comment=[l]{:},
    commentstyle=\color{black},
}

\begin{lstlisting}
{
  "cpoints": [
    {
      "lsb": "00000011",
      "freq_unit": "hz",
      "t_unit": "ns",
      "type": "C",
      "msb": "00000011",
      "time": "10",
      "phase": "0",
      "freq": "100",
      "data": "0",
      "id": 1535931832181
    }
  ],
  "settings": {
    "a_times": "3",
    "a_name": "exp 1",
    "a_channel": "3",
    "a_description": "descr exp 1",
    "a_freq": "100",
    "a_msb": "00000000",
    "a_freq_unit": "hz",
    "a_ts_unit": "ns",
    "a_lsb": "00000000",
    "a_ts": "100",
    "a_bloq": "1",
    "a_phase": "0"
  },
  "execute": true
}
\end{lstlisting}

\newpage

