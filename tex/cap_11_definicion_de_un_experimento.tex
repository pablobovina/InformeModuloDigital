\section{Definicion de un experimento}

En la actualidad una numerosa cantidad de proyectos utilizan notacion JSON para representar
objetos por muchas razones entre ellas su facilidad para ser comprendido por parte de desarrolladores
y el soporte built-in para ser manipulado por navegadores web y nuestros lenguajes elegidos para
desarrollar, python y javascript. Idealmente estos objetos son tambien sencillos de almacenar en 
bases de datos no relacionales.

Haciendo uso de aritmetica podemos deducir que la representacion de un experimento
es liviana puesto que el numero de instrucciones de un programa \(P\) sera de \(N < 512 \) instrucciones
lo que nos daria un peso aproximado en 50KB por experimento representado en esta notacion.

La validacion del esquema JSON de un experimento, sera llevada a cabo por un modulo programado
dentro del sistema. 

Cabe destacar que los tipos de datos presentes en la especificacion de un experiemento son
soportados sin perdidad de precision por parte de la notacion JSON.\cite{json_ref}

\newpage

