\section{Planificacion del desarrollo}


De acuerdo con los requerimientos descriptos en el capitulo XX se propone implementar el sistema
con:

    * Django framework - Pythton 2.7
    * React js - Javascript ECM6
    * Git

Las tecnologias elegidas son ampliamente soportadas por la comunidad de desarrolladores, de codiglo libre y soporte multiplataforma en PC.

Con el objetivo de asegurar que lo primero que se desarrolle cumpla son los requeriminetos asociados 
con el control del Modulo Digital, luego proveer una manipulacion remota y finalmente una interfaz de
usuario para el manejo a traves del navegador web, se organizo el desarrollo de la siguiente manera:

    * Modulo Digital
    * Servidor 
    * Interfaz Grafica Web

El mapeo entre los requerimentos y el orden propuesto genera el siguiente esquema de tareas a realizar.

    * Modulo Digital
        - Configuracion del proyecto y entornos
        - Implementacion control AD 
        - Implementacion control DDS2
        - Implementacion control PP2
        - Implementacion abstraccion Experimento
        - Implementacion de interfaz USB
        - Implementacion de un programa principal
        - Gestion de logs
        - Integracion con Servidor

    * Servidor
        - Configuracion del proyecto y entornos
        - Servicios REST para la edicion de experimentos
        - Servicios REST para control de ejecucion de experimentos
        - Servicios REST para la generacion de reportes
        - Modelado de Base de datos 
        - Configuracion de sesiones
        - Configuracion de acceso a los servicios REST
        - Gestion de logs
        - Integracion con Interfaz WEB
        - Integracion con Modulo Digital
    
    * Interfaz WEB
        - Configuracion de proyecto y entornos
        - Implementacion de transiciones
        - Componente de edicion de experimentos
        - Componente de control de ejecucion de experimentos
        - Componente de login
        - Componente de visualizacion de errores
        - Componente de definicion de experimento
        - Implementacion de control de estados
        - Integracion con Servidor
