\section{Ejecucion de un experimento}

Esta seccion describe los conceptos dry-run main-run y algunos escenarios de ejecucion de
experimentos cubiertos por la implementacion.


Un modo de capturar fallos inesperados es simular la ejecucion de un experimento evitando 
la interaccion con el modulo digital via usb configurando el sistema en modo simulado.

En modo simulado el experimento se ejecuta en un entorno controlado con el objetivo detectar
escencialmente 3 situaciones no deseadas:

* una exepcion de software no capturada
* parametros fuera de rango para alguno de los submodulos
* un error en la traduccion del experimento a un programa ejecutable en el PP2

Este experimento simulado no se ejecuta en un hilo separado puesto que la demora es baja.

De manera alternativa el modo simulado puede verse tambien como un test de integracion
entre las unidades de software desarrolladas.

Si el experimento simulado tiene exito entonces se procede con la ejecucion efectiva del experimento
el cual debemos tener algunos detalles en cuenta:

* puede durar horas
* puede ser cancelado en cualquier momento
* debe estar sincronizado con la base de datos

El siguiente diagrama clases muestra el diseño de un experimento simulado y un experimento.
