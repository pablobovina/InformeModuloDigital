\section{Ejecucion de un experimento}

Un modo de capturar fallos inesperados es simular la ejecucion de un experimento,
evitando la interaccion con el modulo digital via usb.

En modo simulado el experimento se ejecuta en un entorno limitado con el objetivo de detectar 
tres situaciones no deseadas:

\begin{itemize}
\item una exepcion de software no capturada.
\item parametros fuera de rango para alguno de los submodulos.
\item un error en la traduccion del experimento a un programa ejecutable en el PP2.
\end{itemize}

Este experimento simulado no se ejecuta en un hilo separado puesto que la demora es baja.

De manera alternativa el modo simulado puede verse tambien como un test de integracion
entre las unidades de software desarrolladas.

Si el experimento simulado tiene exito entonces se procede con la ejecucion efectiva del experimento
el cual debemos tener algunos detalles en cuenta:

\begin{itemize}
    \item puede durar horas.
    \item puede ser cancelado en cualquier momento.
    \item debe estar sincronizado con la base de datos.
\end{itemize}

El siguiente diagrama clases muestra el diseño de un experimento simulado y un experimento.

\newpage