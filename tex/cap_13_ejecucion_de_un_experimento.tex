\section{Ejecuci\'on de un experimento}

Un modo de capturar fallos es simular la ejecuci\'on de un experimento,
evitando la interacci\'on v\'ia USB, con el m\'odulo digital.
En modo simulado el experimento se ejecuta en un entorno limitado con el objetivo de
detectar tres situaciones no deseadas:

\begin{itemize}
\item Una excepci\'on de software no capturada.
\item Par\'ametros fuera de rango para alguno de los subm\'odulos.
\item Un error en la traducci\'on del experimento a un programa ejecutable en el PP2.
\end{itemize}

Un experimento simulado se ejecuta el hilo principal, puesto que 
la demora es baja. De manera alternativa el modo simulado puede verse tambi\'en como un 
test de integraci\'on entre las unidades de software desarrolladas.
Si el experimento simulado tiene \'exito entonces se procede con la ejecuci\'on efectiva
del experimento que se desprende del hilo principal y en el cual debemos tener 
algunos detalles en cuenta:

\begin{itemize}
    \item Puede durar horas.
    \item Puede ser interrumpido por el usuario en cualquier momento.
    \item Debe estar sincronizado con la base de datos.
\end{itemize}
\newpage