\section{Marco de trabajo e investigacion}

En el proceso de investigacion del fenomeno fisico de Resonancia Magnetica Nuclear el investigador manipula modulos 
electronicos digitales de medicion precisos, estables y en algunos casos si intervienen mas de uno a la vez entre ellos sincronizados 
por medio de interfaces digitales. Estos modulos electronicos digitales colaboran con la creacion del contexto necesario para investigar lo planeado 
segun la necesidad del investigador. En general junto con los modulos electronicos intervienen otros modulos electronicos de naturaleza analogica
tales como amplificadores operacionales, mezcladores de señales, filtros pasa bajos entre otros.

La correcta conexion entre los diferentes modulos electronicos digitales, analogicos y su configuracion durante el proceso de experimentacion
son responsabilidad del investigador y una tarea de suma importancia para el exito de la experiencia a realizar.

En este marco de responsabilidades del investigador y el modulo digital a ser utilizado es deseable y necesaria mecanismos sencillos de manipulacion
del mismo y su monitoreo.

\subsection{Rol del Software}

El rol primario del software en el contexto descripto previamente es el de manipular el modulo digital a traves de la PC utilizada por el investigador 
de forma local o remota.

El rol secundario la administracion de usuarios del modulo digital, monitoreo de los experimentos en curso y resultados.

\newpage

