\section{Marco de trabajo}

En el proceso de investigaci\'on del fenomeno f\'isico de Resonancia Magn\'etica Cuadrupolar y Nuclear 
el investigador manipula m\'odulos electr\'onicos digitales de medici\'on precisos 
y estables, en algunos casos si intervienen mas de uno a la vez, entre ellos sincronizados 
por medio de interfaces digitales. 
Estos m\'odulos electr\'onicos digitales colaboran con la creaci\'on del contexto necesario 
para investigar lo planeado seg\'un la necesidad del investigador.
En general junto con los m\'odulos electr\'onicos digitales intervienen otros m\'odulos electr\'ronicos de 
naturaleza anal\'ogica tales como amplificadores operacionales, mezcladores de señales, 
filtros pasa bajos, entre otros.

La correcta conexi\'on entre los diferentes m\'odulos electr\'onicos digitales, anal\'ogicos y su configuraci\'on durante el proceso de experimentaci\'on son responsabilidad del investigador y una tarea de suma importancia para el \'exito de la experiencia a realizar.

En este marco de responsabilidades del investigador y el m\'odulo digital a ser utilizado es deseable 
y necesario el desarrollo de mecanismos sencillos de manipulaci\'on del mismo y su monitoreo.

\subsection{Rol del Software}

El rol primario del software en el contexto descripto previamente es el de manipular 
el m\'odulo digital a trav\'es de la PC utilizada por el investigador de forma local o remota.

El rol secundario la administraci\'on de usuarios del m\'odulo digital, monitoreo de los experimentos en curso y resultados.

\newpage

