\section{Integraci\'on de los subm\'odulos}

La integraci\'on de los m\'odulos del sistema se implementa a diferentes niveles.

\subsection{Repositorio}

Un subm\'odulo git es un repositorio dentro de otro que lo incluye, con su propio historial de cambios. Cuando hay un cambio en un subm\'odulo el contenedor lo contemplar\'a siempre y cuando lo agregue al historial de este
\cite{git_submodules}.
El repositorio afectado por esta metodolog\'ia es el del \textit{Servidor} incluyento al \textit{M\'odulo Digital} y a la \textit{Interfaz Gr\'afica}.
\\
\\
El \'arbol de directorios del repositorio queda as\'i:

\dirtree{%
.1 Server.
.2 build.
.3 static.
.2 error.
.2 experiments.
.3 M\'odulo Digital.
.2 log.
.2 login.
.2 out.
.2 Interfaz Gr\'afica.
.2 settings.
}

\subsection{Codigo}
El \textit{Servidor} provee el middleware \textit{ManagerThread} para poder interactuar con el \textit{M\'odulo Digital} 
y con el modelo de la base de datos correspondiente seg\'un el estado de la ejecuci\'on del experimento en curso.
Para integrar el \textit{Servidor} con la \textit{Interfaz Gr\'afica} el soporte es built-in via configuraci\'on \textit{Django} el cual provee autenticaci\'on de los servicios siempre y cuando este presente el CSRF-Token en el archivo principal de la \textit{Interfax Gr\'afica}\cite{django_csrf} junto con la cookie de sesi\'on correspondiente en cada petici\'on HTTP\cite{django_cookie}.
\newpage

