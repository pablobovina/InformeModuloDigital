\section{Partes de un experimento}

Existen ciertos elementos tanto humanos como materiales en la realizacion de un experimento junto con un 
simple aunque muy importante uso de datos, es decir que cada uno de estos elementos interactuan compartiendo
informacion que pueden o no determinar acciones.

\subsection{Tipo de Experiencia}
El tipo de experiencia a realizar esta determinada por el investigador y/o area, junto a parametros de planificacion
como por ejemplo la duracion de la misma, la materia a estimular, el tipo de espectromentro, frecuencias
de estimulacion, numero de instantaneas por periodos especificos, tiempos de estimulacion, relajacion, estabilizacion.

\subsection{Sincronizacion}
Algunas experiencias requieren o se complementan con la sincronia entre mudulos digitales de medicion via interfaces 
digitales.