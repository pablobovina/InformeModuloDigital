\section{Partes de un experimento}

Estos son algunos conceptos clave para comprender el contexto y definicion de un
experimento, los cuales se tuvieron en cuenta al momento de la implementacion 
del sistema.

\subsubsection{Pulso de radio frecuencia}
Es una señal senoidal que tiene como objetivo
estimular la muestra presente en el resonador
y que puede ser manipulada previamente
por otros modulos externos. Tiene los siguientes atributos:
    \begin{itemize}
        \item Frecuencia: 0 a 120 megahercios.
        \item Fase: 0 a 360 grados.
        \item Duracion: 0 a 16 segundos.
    \end{itemize}

\subsubsection{Pulso TTL}
Es una señal digital de amplitud 5V y duración predefinida con la finalidad de trasmitirla como entrada a otros aparatos digitales o analogicos para sicronizar momentos de relajacion y estimulacion de las muestras durante la experiencia.

\subsubsection{Muestras}
Las muestras son un grupo de datos obtenidos a cierta frecuenta de muestreo y agrupados de manera contigua 
en un bloque de longitud $2^{n}$ con $n \in \mathbb{N}$. 

\subsection{Tipo de Experiencia}
El tipo de experiencia a realizar esta determinada por el investigador, existen 2 bien difrenciadas:

\subsubsection{Experiencia Promedio}
Una experiencia promedio es una que se repite y donde los bloques de muestras ordenadas se suman uno a uno obteniendo
una mejor representacion los datos para su analisis.

\subsubsection{Experiencia Promedio con Pulso variable}
Es una experiencia promedio donde la configuracion de los pulsos cambia en las sucesivas iteraciones de la misma.
Los pulsos cambian su configuracion para suprimir interferencias de estimulaciones previas y obtener mediciones
mas precisas. Los atributos del pulso que pueden cambiar en las sucesivas iteraciones son la fase y duracion. 

\newpage