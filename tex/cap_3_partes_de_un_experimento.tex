\section{Partes de un experimento}

Estos son algunos conceptos clave para comprender el contexto y definici\'on de un
experimento, los cuales se tuvieron en cuenta al momento de la implementaci\'on 
del sistema.

\subsubsection{Radio frecuencia}
Es una se\~nal senoidal que tiene como objetivo
estimular la muestra presente en el resonador
y que puede ser manipulada previamente
por otros m\'odulos externos. Tiene los siguientes atributos:
    \begin{itemize}
        \item Frecuencia: 0 a 80 megahercios.
        \item Fase: 0 a 360 grados.
        \item Duraci\'on: 0 a 16 segundos.
    \end{itemize}

\subsubsection{Pulso TTL}
Es una se\~nal digital de 5 voltios de amplitud y duraci\'on predefinida con la finalidad de trasmitir como entrada a otros m\'odulos digitales o anal\'ogicos para sincronizar momentos de relajaci\'on y estimulaci\'on de las muestras durante la experiencia.

\subsubsection{Muestras}
Es un grupo de datos obtenidos a cierta frecuencia de muestreo y agrupados de manera contigua 
en un bloque de longitud $2^{n}$ con $n \in \mathbb{N}$. 

\subsection{Tipo de Experiencia}
El tipo de experiencia a realizar esta determinada por el investigador, existen 2 bien diferenciadas:

\subsubsection{Experiencia Promedio}
Una experiencia promedio es una que se repite y donde los bloques de muestras ordenadas se suman uno a uno obteniendo
una mejor representaci\'on los datos para su an\'alisis.

\subsubsection{Experiencia Promedio con Pulso variable}
Es una experiencia promedio donde la configuraci\'on de los pulsos cambia en las sucesivas iteraciones de la misma.
Los pulsos cambian su configuraci\'on para suprimir interferencias de estimulaciones previas y obtener mediciones
mas precisas. Los atributos del pulso que pueden cambiar en las sucesivas iteraciones son la fase y duraci\'on. 

\newpage