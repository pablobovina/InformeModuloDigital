\section{Conclusi\'on}

Dado el desarrollo realizado, los problemas encontrados y los objetivos finales en relaci\'on
al control del m\'odulo digital con la PC via un servidor web, podemos concluir que:

\begin{itemize}
    \item El sistema es capaz del controlar remotamente el \textit{M\'odulo Digital}.
    \item La administraci\'on de experimentos es efectiva.
    \item Actualmente la implementaci\'on se ejecuta en Windows pero para dar soporte a nuevas funcionalidades
    tales como graficaci\'on el n\'umero de bibliotecas de acceso libre predominante da unicamente soporte a Linux.
    \item El desarrollo esta hecho en Python 2.7 pero la \'ultima version de Django 2.x requiere Python 3.x.
    \item El soporte USB por parte de Microchip es \'unicamente a Windows, esto es un cuello de botella a resolver
    por medio del desarrollo de una implementaci\'on del driver con soporte a Linux o por medio de un cambio 
    de arquitectura en el sistema desarrollado.
\end{itemize}

La factibilidad de las 4 primeras propuestas es alta, puesto que el proyecto est\'a definido en 3 partes
bien diferenciadas y desacopladas, aunque es posible que sea necesario sumar funcionalidad 
al \textit{M\'odulo Digital} para tal fin, puesto que ahora ser\'ia un cliente HTTP del \textit{Servidor} y
no un subproceso como se lo define en la arquitectura original.

\newpage
\mbox{}
\newpage

\section{Acr\'onimos}

\begin{itemize}
    \item DDS2: Generador digital de se\~nales.
    \item AD: Conversor anal\'ogico digital.
    \item PP2: Programador de pulsos digitales.
    \item RF: Radio frecuencia.
    \item TTL: Transistor–transistor logic.
    \item HTTP: Hypertext Transfer Protocol.
\end{itemize}
\newpage
\mbox{}
\newpage