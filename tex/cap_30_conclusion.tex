\section{Conlusi\'on}

Dado el desarrollo realizado, los problemas encontrados y los objetivos finales en relaci\'on
al control del m\'odulo digital con la PC via un servidor web, podemos concluir que:

\begin{itemize}
    \item El sistema es capaz del controlar remotamente el \textit{M\'odulo Digital}.
    \item La administraci\'on de experimentos es efectiva.
    \item Es deseable la migraci\'on del Servidor a Linux para un mejor soporte de librerias.
    \item Es deseable la migraci\'on a Python 3.x para mejor soporte y funcionalidades.
    \item El soporte USB unicamente a Windows es una falta de soporte a resolver.
\end{itemize}

La factibilidad de estos cambios es alta, puesto que el proyecto esta definido en 3 partes
bien diferenciadas y desacopladas, aunque es posible que sea necesario sumar funcionalidad 
al \textit{M\'odulo Digital} para tal fin, puesto que ahora seria un cliente HTTP del \textit{Servidor} y
no un subproceso como se lo define en la arquitectura original.

\newpage

\section{Acr\'onimos}

\begin{itemize}
    \item DDS2: Generador digital de señales.
    \item AD: Conversor analo\'gico digital.
    \item PP2: Programador de pulsos digitales.
    \item RF: Radio frecuencia.
    \item TTL: Transistor–transistor logic.
    \item HTTP: Hypertext Transfer Protocol.
\end{itemize}

\newpage