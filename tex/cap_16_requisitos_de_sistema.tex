\section{Requisitos del sistema}

Estos son los requerimientos generales del sistema a desarrollar:

\begin{itemize}
\item Permitir a m\'as de un usuario utilizar el sistema.
\item Permitir a usuarios la gesti\'on de experimentos.
\item Modelar un experimento y su ejecuci\'on.
\item Proveer acceso remoto al sistema.
\end{itemize}
\noindent
De estos requerimientos generales se desprenden los siguientes espec\'ificos:

\begin{itemize}
\item El usuario tiene una sesi\'on asignada al ingreso del sistema y revocada a la salida del mismo.
\item Varios usuarios pueden acceder al sistema al mismo tiempo.
\end{itemize}

\begin{itemize}
\item El usuario puede ver si hay un experimento en ejecuci\'on y el detalle del mismo.
\item El usuario tiene un espacio de trabajo donde puede crear/ver/actualizar/eliminar sus experimentos.
\item El usuario puede cancelar el experimento en ejecuci\'on.
\item El usuario puede solicitar un reporte parcial de un experimento en ejecuci\'on.
\item El usuario puede solicitar el reporte final de un experimento finalizada la ejecuci\'on.
\item El usuario puede verificar el estado de un experimento.
\item El usuario es notificado antes de la ejecuci\'on de un experimento en las siguientes situaciones:
    \begin{itemize}
    \item Salida voluntaria por error:
        \begin{itemize}
        \item No es una secuencia ejecutable por el PP2.
        \item Ya existe alg\'un experimento ejecut\'andose.
        \item Alg\'un par\'ametro de configuraci\'on es in\'valido en alg\'un submodulo PP2, DDS2, AD, USB.
        \item Fall\'o en conexi\'on v\'ia USB.
        \end{itemize}
    \end{itemize}    
    \begin{itemize}    
    \item Salida involuntaria por error:
        \begin{itemize}
        \item Hubo alguna excepci\'on de sistema no controlada.
        \item No se pudo establecer conexi\'on con el servidor.
        \item Un experimento finalizado no actualiz\'o su estado impidiendo la ejecuci\'on de otro.
        \end{itemize}
    \end{itemize}
\end{itemize}
\begin{itemize}
\item Un experimento tiene una secuencia definida con sus par\'ametros.
\item Un experimento tiene una marca de tiempo de creaci\'on/actualizaci\'on.
\item Un experimento eliminado no es recuperable.
\item Un experimento tiene un autor que es el usuario que lo creo.
\item Un experimento tiene estado $Created$ cuando est\'a almacenado en base de datos.
\item Un experimento es solo visible para el usuario que lo creo.
\item Un experimento tiene un t\'itulo.
\item Un experimento tiene una descripci\'on.
\item Un experimento no tiene un historial de edici\'on asociado.
\end{itemize}

\begin{itemize}
\item Un resultado es el producto de la ejecuci\'on de un experimento.
\item Un resultado es parcial cuando la ejecuci\'on del experimento asociado no finaliz\'o.
\item Un resultado es final cuando la ejecuci\'on del experimento asociado finaliz\'o.
\item Un resultado tiene un \'unico experimento asociado.
\end{itemize}

\begin{itemize}
\item Un reporte parcial es el grupo de datos de un resultado parcial.
\item Un reporte final es el grupo de un resultado final.
\item Un reporte contiene:
    \begin{itemize}
    \item Log de la ejecuci\'on.
    \item Datos del AD en formato CSV.
    \item Experimento ejecutado.
    \end{itemize}
\end{itemize}
\begin{itemize}
\item El sistema tiene servicios REST para:
    \begin{itemize}
    \item Crear/Ver/Editar/Eliminar experimentos.
    \item Iniciar/Cancelar ejecuci\'on de un experimento.
    \item Cancelar la ejecuci\'on de todos los experimentos.
    \item Descargar reportes parciales y finales.
    \item Proveer autenticaci\'on a todos los servicios.
    \end{itemize}
\end{itemize}
\begin{itemize}
\item El sistema tiene una interfaz de usuario web.
\end{itemize}
\newpage