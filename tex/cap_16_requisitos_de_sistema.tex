\section{Requisitos del sistema}

Podemos ver los requerimientos en generales

\begin{itemize}
\item permitir a mas de un usuario utilizar el sistema
\item permitir a usuarios la gestion de experimentos
\item modelar un experimento y su ejecucion
\item proveer acceso remoto al sistema
\end{itemize}
De estos requerimientos se desprenden los siguientes especificos:

\begin{itemize}
\item el usuario tiene una sesion asignada al ingreso del sistema y revocada a la salida del mismo.
\item varios usuarios pueden acceder al sistema al mismo tiempo
\end{itemize}

\begin{itemize}
\item el usuario puede ver si hay un experimento en ejecucion y el detalle del mismo
\item el usuario tiene un espacio de trabajo donde puede crear/ver/actualizar/eliminar sus experimentos
\item el usuario puede cancelar el experimento en ejecucion
\item el usuario puede solicitar un reporte parcial de un experimento en ejecucion
\item el usuario puede solicitar el reporte final de un experimento finalizada la ejecucion
\item el usuario puede verificar el estado de un experimento:
\item el usuario es notificado antes de la ejecucion de un experimento cuando:
    \begin{itemize}
    \item salida voluntaria por error:
        \begin{itemize}
        \item no es una secuencia ejecutable por el PP2
        \item ya existe algun experimento ejecutandose
        \item algun parametro de configuracion es invalido en algun submodulo PP2, DDS2, AD, USB
        \item fallo en conexion via USB
        \end{itemize}
    \end{itemize}    
    \begin{itemize}    
    \item salida involuntaria por error:
        \begin{itemize}
        \item hubo alguna excepcion de sistema no controlada
        \item no se pudo establecer conexion con el servidor
        \item un experimento finalizado no actualizo su estado impidiendo la ejecucion de solcitido
        \end{itemize}
    \end{itemize}
\end{itemize}
\begin{itemize}
\item un experimento tiene una secuencia definidida con sus parametros
\item un experimento tiene una marca de tiempo de creacion/actualizacion
\item un experimento eliminado no es recuperable
\item un experimento tiene un autor que es el usuario que lo creo
\item un experimento tiene estado created cuando esta almacenado en base de datos
\item un experimento es solo visible para el usuario que lo creo
\item un experimento tiene un titulo
\item un experimento tiene una descripcion
\item un experimento no tiene un historial de edicion asociado
\end{itemize}

\begin{itemize}
\item un resultado es el producto de la ejecucion de un experimento
\item un resultado es parcial cuando la ejecucion del experimento asociado no finalizo
\item un resultado es final cuando la ejecucion del experimento asociado finalizo
\item un resultado tiene un unico experimento asociado
\end{itemize}

\begin{itemize}
\item un reporte parcial es el grupo de datos de un resultado parcial
\item un reporte final es el grupo de un resultado final
\item un reporte contiene:
    \begin{itemize}
    \item log de la ejecucion
    \item datos del AD en formato CSV
    \item experimento ejecutado
    \end{itemize}
\end{itemize}
\begin{itemize}
\item el sitema tiene servicios REST para:
    \begin{itemize}
    \item crear/ver/editar/eliminar experimentos
    \item iniciar/cancelar ejecucion de un experimento
    \item cancelar la ejecucion de todos los experimentos
    \item descargar reportes parciales y finales
    \item proveer autenticacion a todos los servicios
    \end{itemize}
\end{itemize}
\begin{itemize}
\item el sistema tiene una interfaz de usuario web
\end{itemize}
\newpage