\section{Requisitos del sistema}

Podemos ver los requerimientos en generales


* permitir a mas de un usuario utilizar el sistema
* permitir a usuarios la gestion de experimentos
* modelar un experimento y su ejecucion
* proveer acceso remoto al sistema

De estos requerimientos se desprenden los siguientes especificos:

* el usuario tiene una sesion asignada al ingreso del sistema y revocada a la salida del mismo.
* varios usuarios pueden acceder al sistema al mismo tiempo


* el usuario puede ver si hay un experimento en ejecucion y el detalle del mismo
* el usuario tiene un espacio de trabajo donde puede crear/ver/actualizar/eliminar sus experimentos
* el usuario puede cancelar el experimento en ejecucion
* el usuario puede solicitar un reporte parcial de un experimento en ejecucion
* el usuario puede solicitar el reporte final de un experimento finalizada la ejecucion
* el usuario puede verificar el estado de un experimento:
* el usuario es notificado antes de la ejecucion de un experimento cuando:
    * salida voluntaria por error:
        * no es una secuencia ejecutable por el PP2
        * ya existe algun experimento ejecutandose
        * algun parametro de configuracion es invalido en algun submodulo PP2, DDS2, AD, USB
        * fallo en conexion via USB
    * salida involuntaria por error:
        * hubo alguna excepcion de sistema no controlada
        * no se pudo establecer conexion con el servidor
        * un experimento finalizado no actualizo su estado impidiendo la ejecucion de solcitido

* un experimento tiene una secuencia definidida con sus parametros
* un experimento tiene una marca de tiempo de creacion/actualizacion
* un experimento eliminado no es recuperable
* un experimento tiene un autor que es el usuario que lo creo
* un experimento tiene estado created cuando esta almacenado en base de datos
* un experimento es solo visible para el usuario que lo creo
* un experimento tiene un titulo
* un experimento tiene una descripcion
* un experimento no tiene un historial de edicion asociado

* un resultado es el producto de la ejecucion de un experimento
* un resultado es parcial cuando la ejecucion del experimento asociado no finalizo
* un resultado es final cuando la ejecucion del experimento asociado finalizo
* un resultado tiene un unico experimento asociado

* un reporte parcial es el grupo de datos de un resultado parcial
* un reporte final es el grupo de un resultado final
* un reporte contiene:
    * log de la ejecucion
    * datos del AD en formato CSV
    * experimento ejecutado

* el sitema tiene servicios REST para:
    * crear/ver/editar/eliminar experimentos
    * iniciar/cancelar ejecucion de un experimento
    * cancelar la ejecucion de todos los experimentos
    * descargar reportes parciales y finales
    * proveer autenticacion a todos los servicios

* el sistema tiene una interfaz de usuario web