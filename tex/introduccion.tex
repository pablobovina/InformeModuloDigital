\section{Introducci\'on}

El \'area de Resonancia Magn\'etica Cuadrupolar
y Nuclear de FaMAF posee en su inventario un M\'odulo Digital
para el estudio de cristales moleculares que se desarrollan 
en este grupo de investigaci\'on, el cual no dispone de una 
interfaz de uso manual, puesto que est\'a pensado para el 
uso via interfaz USB con la PC y actualmente en fuera 
de uso por la falta de la misma.

En el presente trabajo se aborda el desarrollo de software para
el control del M\'odulo Digital y de manera complementaria una
plataforma web que colabore con la gesti\'on de los experimentos 
vinculados con el uso del mismo.

El objetivo de proveer una plataforma web es la de brindar el acceso 
a los experimentos y sus resultados de manera remota junto con 
la capacidad de realizar experiencias de investigaci\'on desde la oficina
con una ayuda m\'inima de alg\'un colaborador del area presente cerca
de los dem\'as equipos, puesto que siempre hay un miembro del \'area presente
por razones de seguridad.

El Módulo Digital consta de tres submódulos:

\begin{itemize}

   \item Programador de pulsos digitales: tiene la finalidad 
   de ejecutar una secuencia de pulsos para coordinar los 
   subm\'odulos del M\'odulo Digital.

   \item Generador digital de se\~nales: genera las se\~nales de 
   estimulaci\'on a los n\'ucleos resonantes de la materia a investigar.

   \item Conversor anal\'ogico digital: convierte las 
   se\~nales anal\'ogicas de la resonancia a digitales
   para su posterior an\'alisis.

\end{itemize}

Cada uno de estos subm\'odulos posee su propia arquitectura y comandos 
para su manipulaci\'on los cuales ser\'an partes claves en el desarrollo del 
software y la integraci\'on de los mismos en un \'unico controlador.

Existen varios tipos de experiencias posibles de ser modeladas 
con el uso del M\'odulo digital entre ellas dos:

\begin{itemize}
    \item Experiencia Promedio: una secuencia iterativa de 
    pulsos de radio frecuencia de propiedades fijas.

    \item Experiencia Promedio con Pulso variable: una 
    experiencia promedio donde los pulsos de radio frecuencia 
    varian sus propiedades durante las iteraciones de la misma.

\end{itemize}

El objetivo es poder modelar ambas aunque con la \textit{Experiencia Promedio} es 
suficiente para el alcance de este trabajo.


\newpage