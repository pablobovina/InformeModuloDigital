\section{Algoritmo de traduccion de experimentos}

Esta seccion describe como un experimento es traducido a instrucciones que el PP2 pueda almacenar y ejcutar.

Como mencionamos en el capitulo XX un programa almacenable y ejecutable por el PP2 es una secuencia de 
instrucciones I1 .. IN con N no mayor a K, donde K esta relacionado al limite de memoria del mismo es de XX MB.

Las instrucciones disponibles son:

*continue
*loop
*retl
*fin

No todos los programas escritos con estas instrucciones son admitidos como validos para el PP2, 
por esto tenemos las siguientes reglas que deben cumplir los programas ejecutables por el PP2.

Regla 1:
la instruccion fin siempre es la ultima instruccion.

Regla 2:
la intruccion fin aparece solo una vez.

Regla 3:
la intruccion fin siempre esta presente.

Regla 4:
un bloque siempre inicia con instruccion loop y finaliza con retl

Regla 5:
puede haber hasta 3 loops dentro de un loop

Estas reglas definen el siguiente automata:



Resultando en el siguiente algoritmo base: 


traslate(P, L, S)

    ins = next(P)

    if ins = Continue:
        S.append(ins)

    if ins = Retl:
        if(L!=0):
            S.append(ins)
            L--
        else:
            ret Error

    if ins = Loop:
        if(L<4):
            S.append(ins)
            L++
        else:
            ret Error
    
    ret S