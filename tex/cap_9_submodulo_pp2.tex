\section{Submodulo PP2}

El programador de pulsos es un microprocesador diseñado a medida con instrucciones que generan
un patron de salida por un tiempo determinado.
\\\\
Un patron de salida es una combinacion de 16 pulsos TTL en paralelo por un periodo 
de tiempo determinado.
\\\\
Un programa ejecutable por el PP2 sera una secuencia de no mas de 512 instrucciones 
y la ejecucion derivara en una secuencia de pulsos de duracion determinada.

\subsection{Instrucciones}

El PP2 tiene 4 instrucciones basicas: Continue, RetL, Loop, End, 
cada una es una secuencia de 64 bits con la siguiente estrucutura:\\
\begin{table}[ht]
    \centering
    \begin{tabular}{|l|l|l|l|l|l|l|}
    \hline
    Patron de salida  & Dato & Nivel de lazo & Codigo de instruccion & Demora \\
    \hline
    16 bits & 11 bits & 2 bits & 3 bits & 32 bits\\
    \hline
\end{tabular}
\caption{\label{tab:pp2_ins}Estructura de las instrucciones.}
\end{table}


\subsubsection{Duracion de una instruccion}
La base de tiempo es 40 nanosegundos. ($1ns = 1*10^{–9}$ segundos) 
Cada instrucción requiere de 4 pulsos de reloj ($4 * 40ns = 160ns$) 
y la demora mínima es 2 ($2 * 40ns = 80ns$). Por lo tanto el pulso de
valle mínimo es de $160ns + 80ns = 240ns$.

\subsubsection{Continue}
Mantiene una combinacion de 16 pulsos de salida por un tiempo determinado.
\begin{itemize}
    \item patron de salida: es la combinacion de 16 pulsos de salida.
    \item dato: no utilizado.
    \item nivel de lazo: no utilizado.
    \item codigo de instruccion: 0x01.
    \item demora: duracion de la instruccion.
\end{itemize}

\subsubsection{Loop}
Marca el inicio de un bloque de repeticion.
\begin{itemize}
    \item patron de salida: es la combinacion de 16 pulsos de salida.
    \item dato: contador de repeticiones.
    \item nivel de lazo: nivel de anidamiento entre lazos
    \item codigo de instruccion: 0x02.
    \item demora: duracion de la instruccion.
\end{itemize}

\subsubsection{Retl}
Marca de fin de un bloque de repeticion.
\begin{itemize}
    \item patron de salida: es la combinacion de 16 pulsos de salida.
    \item dato: direccion de la instruccion Loop de inicio.
    \item nivel de lazo: no utilizado.
    \item codigo de instruccion: 0x03.
    \item demora: duracion de la instruccion.
\end{itemize}

\subsubsection{End}
Finaliza la secuencia de pulsos.
\begin{itemize}
    \item patron de salida: es la combinacion de 16 pulsos de salida.
    \item dato: no utilizado.
    \item nivel de lazo: no utilizado.
    \item codigo de instruccion: 0x07.
    \item demora: duracion de la instruccion.
\end{itemize}

\subsection{Registros del PP2}

El PP2 cuenta con los siguientes registros de 8 bits:

\begin{table}[ht]
    \centering
    \begin{tabular}{|l|l|l|l|}
    \hline
    Direccion  & Descripcion & modo  \\
    \hline
    0x50 & comando & escritura\\
    \hline
    0x51 & carga   & escritura\\
    \hline
    0x52 & señal   & escritura \\
    \hline
\end{tabular}
\caption{\label{tab:pp2_reg}Estructura de las instrucciones.}
\end{table}
\newpage
\subsubsection{Carga de una instruccion en el PP2.}
\begin{algorithm}
    \caption{Carga de una instruccion en el PP2}\label{algo_pp2_load}
    \begin{algorithmic}[1]
    \Procedure{UploadProgram}{}
    \State // {Reset}
    \State $write(0x50, 0x02)$
    \State // {Modo carga}
    \State $write(0x50, 0x03)$
    \State // {Continue 0x55AA 4}
    \State $write(0x51, 0x04)$
    \State $write(0x51,0x00)$
    \State $write(0x51,0x00)$
    \State $write(0x51,0x00)$
    \State $write(0x51,0x01)$
    \State $write(0x51,0x00)$
    \State $write(0x51,0xAA)$
    \State $write(0x51,0x55)$
    \State $write(0x52, 0x00)$
    \State // {End}
    \State $write(0x51,0x00)$
    \State $write(0x51,0x00)$
    \State $write(0x51,0x00)$
    \State $write(0x51,0x00)$
    \State $write(0x51,0x00)$
    \State $write(0x51,0x00)$
    \State $write(0x51,0x00)$
    \State $write(0x51,0x00)$
    \State $write(0x52,0x00)$
    \EndProcedure
    \end{algorithmic}
    \end{algorithm}

\subsubsection{Ejecutar un programa.}
\begin{algorithm}
    \caption{Ejecutar un programa.}\label{algo_pp2_exe}
    \begin{algorithmic}[1]
    \Procedure{Setup}{}
    \State // {Modo microprocesador}
    \State $write(0x50, 0x00)$
    \State // {Señal de ejecucion}
    \State $write(0x08)$
    \EndProcedure
    \end{algorithmic}
    \end{algorithm}
\newpage
