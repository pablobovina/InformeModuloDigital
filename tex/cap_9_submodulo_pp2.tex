\section{Subm\'odulo PP2}

El programador de pulsos es un microprocesador dise\~nado a medida con instrucciones que generan
un patr\'on de salida por un tiempo determinado.
\\\\
Un patr\'on de salida es una combinaci\'on de 16 pulsos TTL en paralelo por un periodo 
de tiempo determinado.
\\\\
Un programa ejecutable por el PP2 sera una secuencia de no mas de 512 instrucciones 
y la ejecucion derivara en una secuencia de patrones de salida de duracion determinada.


\subsection{Instrucciones}

El PP2 tiene 4 instrucciones b\'asicas: $Continue$, $Retl$, $Loop$ y $End$ 
cada una es una secuencia de 64 bits con la siguiente estrucutura:\\
\begin{table}[ht]
    \centering
    \begin{tabular}{|l|l|l|l|l|l|l|}
    \hline
    Patr\'on de salida  & Dato & Nivel de lazo & Codigo de instrucci\'on & Demora \\
    \hline
    16 bits & 11 bits & 2 bits & 3 bits & 32 bits\\
    \hline
\end{tabular}
\caption{\label{tab:pp2_ins}Estructura de las instrucciones.}
\end{table}


\subsubsection{Duraci\'on de una instrucci\'on}
Cada instrucci\'on requiere de 4 pulsos de reloj ($4 * 40ns = 160ns$) 
y la demora m\'inima es 2 pulsos ($2 * 40ns = 80ns$). Por lo tanto el pulso de
valle m\'inimo es de $160ns + 80ns = 240ns$.

\subsubsection{Continue}
Mantiene una combinaci\'on de 16 pulsos de salida por un tiempo determinado.
\begin{itemize}
    \item patr\'on de salida: es la combinaci\'on de 16 pulsos de salida.
    \item dato: no utilizado.
    \item nivel de lazo: no utilizado.
    \item c\'odigo de instruccion: 0x01.
    \item demora: duraci\'on de la instrucci\'on.
\end{itemize}

\subsubsection{Loop}
Marca el inicio de un bloque de repetici\'on.
\begin{itemize}
    \item patr\'on de salida: es la combinaci\'on de 16 pulsos de salida.
    \item dato: contador de repeticiones.
    \item nivel de lazo: nivel de anidamiento entre lazos
    \item c\'odigo de instrucci\'on: 0x02.
    \item demora: duraci\'on de la instrucci\'on.
\end{itemize}

\subsubsection{Retl}
Marca de fin de un bloque de repetici\'on.
\begin{itemize}
    \item patron de salida: es la combinaci\'on de 16 pulsos de salida.
    \item dato: direcci\'on de la instrucci\'on Loop de inici\'o.
    \item nivel de lazo: no utilizado.
    \item c\'odigo de instrucci\'on: 0x03.
    \item demora: duraci\'on de la instrucci\'on.
\end{itemize}

\subsubsection{End}
Finaliza la secuencia de pulsos.
\begin{itemize}
    \item patr\'on de salida: es la combinaci\'on de 16 pulsos de salida.
    \item dato: no utilizado.
    \item nivel de lazo: no utilizado.
    \item c\'odigo de instrucci\'on: 0x07.
    \item demora: duraci\'on de la instrucci\'on.
\end{itemize}

\subsection{Registros del PP2}

El PP2 cuenta con los siguientes registros de 8 bits:

\begin{table}[ht]
    \centering
    \begin{tabular}{|l|l|l|l|}
    \hline
    Direccion  & Descripcion & modo  \\
    \hline
    0x50 & comando & escritura\\
    \hline
    0x51 & carga   & escritura\\
    \hline
    0x52 & señal   & escritura \\
    \hline
\end{tabular}
\caption{\label{tab:pp2_reg}Estructura de las instrucciones.}
\end{table}
\newpage
\subsubsection{Carga de una instruccion en el PP2.}
\begin{algorithm}
    \caption{Carga de una instruccion en el PP2}\label{algo_pp2_load}
    \begin{algorithmic}[1]
    \Procedure{UploadProgram}{}
    \State // {Reset}
    \State $write(0x50, 0x02)$
    \State // {Modo carga}
    \State $write(0x50, 0x03)$
    \State // {Continue 0x55AA 4}
    \State $write(0x51, 0x04)$
    \State $write(0x51,0x00)$
    \State $write(0x51,0x00)$
    \State $write(0x51,0x00)$
    \State $write(0x51,0x01)$
    \State $write(0x51,0x00)$
    \State $write(0x51,0xAA)$
    \State $write(0x51,0x55)$
    \State $write(0x52, 0x00)$
    \State // {End}
    \State $write(0x51,0x00)$
    \State $write(0x51,0x00)$
    \State $write(0x51,0x00)$
    \State $write(0x51,0x00)$
    \State $write(0x51,0x00)$
    \State $write(0x51,0x00)$
    \State $write(0x51,0x00)$
    \State $write(0x51,0x00)$
    \State $write(0x52,0x00)$
    \EndProcedure
    \end{algorithmic}
    \end{algorithm}

\subsubsection{Ejecutar un programa.}
\begin{algorithm}
    \caption{Ejecutar un programa.}\label{algo_pp2_exe}
    \begin{algorithmic}[1]
    \Procedure{Setup}{}
    \State // {Modo microprocesador}
    \State $write(0x50, 0x00)$
    \State // {Señal de ejecucion}
    \State $write(0x08)$
    \EndProcedure
    \end{algorithmic}
    \end{algorithm}
\newpage
