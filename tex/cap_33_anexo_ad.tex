\section{Anexo C: Comandos del AD}

El conversor anal\'ogico digital tiene 2 modos de trabajo bien definidos, modo PC y modo AD. 
Los modos son elegidos con el bit cero del \textit{Byte de control} en 0 y 1 respectivamente.

En modo PC se pueden realizar 2 acciones principales: la configuraci\'on general y la extracci\'on de los datos.
En modo AD, el conversor se pone en espera del pulso 5 para realizar la conversi\'on de los datos para una posterior extracci\'on de parte del usuario.

La longitud del buffer de adquisici\'on es variable en bloques de $64*16*2^{i}$ Bytes con $0 \leq i \leq 7$, la elecci\'on del bloque se hace con los bits 4,5,6 del yte de control.

Internamente el AD cuenta con contador de direcciones que indexa el buffer de adquisici\'on y el mismo debe ser reiniciado en 2 ocasiones: previo a una adquisici\'on y antes de realizar una extracci\'on de datos.
Esto se hace por medio del bit 7 del btye de control poniendolo sucesivamente en 1 y 0. 

Existen 2 formas de adquisici\'on, por pulsaci\'on y continuo, en este trabajo utilizamos el continuo por 
medio del bit 1 del \textit{Byte de control} en 0.

Los bits 3 y 4 del \textit{Byte de control} son sin cuidado y su valor es puesto en cero.

Es posible determinar el estado de finalizaci\'on de la conversi\'on por medio del comando de estado, 
el cual responde con un Byte donde el valor del bit 0 indicar\'a en cero el fin de la adquisici\'on.

La configuraci\'on del intervalo de muestreo se har\'a almacenando en el registro 0x0C el Byte
de represetaci\'on del tiempo entre muestra y muestra.

La obtenci\'on de los datos del buffer de adquisici\'on requiere el reset del contador (bit 7 en 1 y 0 sucesivamente)
antes de la ejecuci\'on del comando de extraccio\'n de datos, el cual tiene como par\'ametro la direcci\'on del 
buffer y como respuesta un arreglo contiguo de Bytes acorde al n\'umero de bloques preconfigurado en el registro de control 0x0B.

\subsection{Comandos USB}
\begin{itemize}
\item Control: \begin{verbatim}['r'][0x0B][Byte de control]\end{verbatim}
\item Configuraci\'on de intervalo: \begin{verbatim}['t'][0x0C][Byte ts]\end{verbatim}
\item Extracci\'on de adquisiciones: \begin{verbatim}['B'][OxO9|0x0A|0x0B]\end{verbatim}
\item Estado: \begin{verbatim}['S'][0x0B][0x00]\end{verbatim}
\end{itemize}

\newpage