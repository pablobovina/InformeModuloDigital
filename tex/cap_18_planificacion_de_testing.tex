\section{Planificaci\'on de Testing}

La planificaci\'on de testing del sistema tiene varios enfoques en respuesta a diferentes objetivos:
\begin{itemize}
    \item Brindar una experiencia predecible al usuario final.
    \item Lograr una integraci\'on coherente entre m\'odulos.
    \item Aumentar cobertura de requerimientos de usuario y sistema.
    \item Facilitar la comprensi\'on del comportamiento esperado del sistema.
\end{itemize}
Por unidad existen las siguientes validaciones:
\begin{itemize}
    \item M\'odulo Digital
    \begin{itemize}
        \item Entradas de configuraci\'on para el PP2, DDS2 y AD.
        \item Entradas de programa ejecutable por el PP2.
        \item Existencia de conexi\'on via interfaz USB durante la ejecuci\'on de experimentos.
        \item Salidas de se\~nales RF.
        \item Salidas de pulsos TTL.
        \item Input de conversi\'on de datos del AD.
        \item Salida de reportes de los canales A y B.
    \end{itemize}
    \item Servidor
    \begin{itemize}
        \item Autenticaci\'on obligatoria en los servicios.
        \item Ejecuci\'on secuencial de los experimentos.
        \item Aislamiento de espacio de trabajo de usuarios.
        \item Coherencia en los estados de los experimentos.
        \item Generaci\'on de reportes.
        \item Gesti\'on de procesos de ejecuci\'on de experimentos.
    \end{itemize}
    \item Interfaz de usuario web
    \begin{itemize}
        \item Visualizaci\'on coherente de elementos y estilos.
        \item Tiempos cortos de carga de las vistas.
        \item Visualizaci\'on de alertas de usuario.
        \item Transiciones entre vistas.
    \end{itemize}
\end{itemize}
Dados los diferentes enfoques son necesarias se eligieron las siguientes para probar:
\begin{itemize}
    \item \textit{Postman} para simular comunicaci\'on HTTP con servicios REST.
    \item \textit{Arduino} para visualizar los pulsos TTL via \textit{Serial Plotter}.
    \item \textit{Osciloscopio} para medir se\~nales RF generadas en los experimentos.
    \item \textit{Generador de se\~nales} para simular adquisici\'on de datos del AD.
    \item \textit{Google Chrome} y herramientas para validar estados de la interfaz grafica.
    \item \textit{Mock del servidor} escrito en \textit{Flask-Python} para simular servicios REST.
\end{itemize}
\newpage