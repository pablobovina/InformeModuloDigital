\section{Planificacion de Testing}

La planificacion de testing del sistema tiene varios enfoques en respuesta a diferentes objetivos:

    * brindar una experiencia predecible y agradable al usuario final
    * lograr una integracion coherente entre modulos
    * aumentar cobertura de requerimientos de usuario y sistema
    * facilitar la comprension del comportamiento esperado del sistema

Por unidad existen las siguientes validaciones:

    Modulo Digital
        
        * entradas de configuracion para el PP2, DDS2 y AD
        * entradas de programa ejecutable por el PP2
        * existencia de conexion via interfaz USB durante la ejecucion de experimentos
        * salidas de señales RF
        * salidas de pulsos TTL
        * input de conversion de datos del AD
        * salida de reportes de los canales A y B.

    Servidor

        * autenticacion obligatoria en los servicios
        * ejecucion secuencial de los experimentos
        * aislamiento de espacio de trabajo de usuarios
        * coherencia en los estados de los experimentos
        * generacion de reportes
        * gestion de procesos de ejecucion de experimentos

    Interfaz de usuario web

        * visualizacion coherente de elementos y estilos
        * tiempos cortos de carga de las vistas
        * visualizacion de alertas de usuario
        * transiciones entre vistas

Dados los diferentes enfoques son necesarias se eligieron las siguientes para probar:

    * Postman para simular comunicacion http con servicios REST
    * Arduino para visualizar los pulsos TTL via Serial Plotter
    * Osciloscopio para medir señales RF generadas en los experimentos.
    * Generador de señales para simular adquisicion de datos del AD.
    * Google Chrome y herramientas para validar estados de la interfaz grafica.
    * Mock del servidor escrito en Flask-Python para simular servicios REST.

    \newpage