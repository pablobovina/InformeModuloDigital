\section{Planificacion de Testing}

La planificacion de testing del sistema tiene varios enfoques en respuesta a diferentes objetivos:
\begin{itemize}
    \item brindar una experiencia predecible y agradable al usuario final
    \item lograr una integracion coherente entre modulos
    \item aumentar cobertura de requerimientos de usuario y sistema
    \item facilitar la comprension del comportamiento esperado del sistema
\end{itemize}
Por unidad existen las siguientes validaciones:
\begin{itemize}
    \item Modulo Digital
    \begin{itemize}
        \item entradas de configuracion para el PP2, DDS2 y AD
        \item entradas de programa ejecutable por el PP2
        \item existencia de conexion via interfaz USB durante la ejecucion de experimentos
        \item salidas de señales RF
        \item salidas de pulsos TTL
        \item input de conversion de datos del AD
        \item salida de reportes de los canales A y B.
    \end{itemize}
    \item Servidor
    \begin{itemize}
        \item autenticacion obligatoria en los servicios
        \item ejecucion secuencial de los experimentos
        \item aislamiento de espacio de trabajo de usuarios
        \item coherencia en los estados de los experimentos
        \item generacion de reportes
        \item gestion de procesos de ejecucion de experimentos
    \end{itemize}
    \item Interfaz de usuario web
    \begin{itemize}
        \item visualizacion coherente de elementos y estilos
        \item tiempos cortos de carga de las vistas
        \item visualizacion de alertas de usuario
        \item transiciones entre vistas
    \end{itemize}
\end{itemize}
Dados los diferentes enfoques son necesarias se eligieron las siguientes para probar:
\begin{itemize}
    \item Postman para simular comunicacion http con servicios REST
    \item Arduino para visualizar los pulsos TTL via Serial Plotter
    \item Osciloscopio para medir señales RF generadas en los experimentos.
    \item Generador de señales para simular adquisicion de datos del AD.
    \item Google Chrome y herramientas para validar estados de la interfaz grafica.
    \item Mock del servidor escrito en Flask-Python para simular servicios REST.
\end{itemize}
\newpage